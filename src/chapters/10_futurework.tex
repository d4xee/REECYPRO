\chapter{Future Work\authorB{}}

To further optimize this endeavor for industrial applicability, several imperative steps must be
undertaken. Primarily, there is a critical need to augment the efficiency of bacterial proliferation and
the resultant yield of Rare Earth Elements (REEs). Achieving this entails the development of
innovative methodologies aimed at curtailing the growth duration and nutrient utilization by
Methylorubrum Extorquens. Moreover, the primary substrate for bacterial metabolism, methanol,
could be replaced with its more rudimentary and economically advantageous precursor, methane,
which is also amenable to metabolization by M. Extorquens. To realize the full-scale industrialization
of this project, rigorous testing in expansive bioreactor systems is indispensable, alongside the
incorporation of diverse forms of Electronic Waste.
Determining the most economically viable category of E-Waste necessitates extensive experimental
evaluation. Concurrently, alongside the E-Waste assessments, there arises a pressing need for the
development of a high-capacity, efficient, and durable shredding apparatus tailored to handle various
types of E-Waste. This undertaking poses multifaceted challenges, particularly in terms of safety and
cost considerations. An industrial-grade E-Waste shredder must be inherently non-combustible and
proficient in processing metal, plastic, fiberglass, and adhesive materials, all while maintaining
optimal power consumption levels and facilitating facile maintenance protocols.