\chapter{Experimental Section}

In the following chapter, we describe the work we carried out and why we did it.
We begin with methods for cultivating bacteria, which are followed by the extraction of proteins from \emph{M. extorquens}.
Afterwards, we describe how we analyzed the proteins.
In the end, we show you how we did the arsenazo-III assay.

\section{Cultivation of Bacteria\authorB}

Preparation of solid nutrient solution for petri dishes:

\textbf{Materials:}

\begin{itemize}
    \item Peptone 2,5g
    \item Meat Extract 1,5g
    \item Agar 7,5g
    \item \ce{H2O} 500mL
    \item Scale
    \item Autoclave bottle
    \item Spatula
\end{itemize}


\textbf{Execution:}
\begin{itemize}
    \item Weigh all the necessary ingredients
    \item Fill the autoclave bottle with around 100ml of water
    \item Add the dry ingredients to the autoclave bottle
    \item Mix the dry ingredients with water
    \item Add the remaining water to the bottle
    \item Shake until mixed
\end{itemize}


Finalizing the solid nutrient solution to be poured into petri dishes:

\textbf{Materials:}

\begin{itemize}
    \item Petri dishes
    \item Autoclave
    \item Autoclave indicator tape
    \item Sterile workbench
\end{itemize}


\textbf{Execution:}

\begin{itemize}
    \item Add a piece of indicator tape on the cap of the autoclave bottle
    \item Autoclave the nutrient solution at 121°C and 1 bar for 15 minutes
    \item After autoclaving the nutrient solution swiftly pour it into the petri dishes
    \item Leave the nutrient solution to harden
    \item Flip the petri dishes carefully on their cover and put them in the fridge
\end{itemize}



\section{Protein Extraction\authorB}

\section{Protein Analysis\authorA}

\subsection{IR-Spectrometry}

The IR-Spectrometry was done using an ATR-IR-Spectrometer.
This meant that we could place the sample directly onto the device, without additional sample preparation.

The following table gives an overview of the lysed samples of \emph{M. extorquens} we measured, what treatments we gave the bacteria to grow in and when we measured the samples after the treatment.

\begin{table}[H]
    \begin{tabularx}{\textwidth}{l p{5cm} X}
        \hline
        \textbf{Type of sample} & \textbf{Treatment} & \textbf{Weeks of growth after treatment} \\ \hline
        Proteins dissolved in water & no treatment & 0, 1 and 2 \\
        %Proteins dissolved in water & no treatment & 1 \\
        %Proteins dissolved in water & no treatment & 2 \\
        Cell paste & no treatment & 0, 1 and 2 \\
        %Cell paste & no treatment & 1 \\
        %Cell paste & no treatment & 2 \\
        Proteins dissolved in water & 0,5mL of \ce{Ce}-solution & 0, 1 and 2 \\
        %Proteins dissolved in water & 0,5mL of \ce{Ce}-solution & 1 \\
        %Proteins dissolved in water & 0,5mL of \ce{Ce}-solution & 2 \\
        Cell paste & 0,5mL of \ce{Ce}-solution & 0, 1 and 2 \\
        %Cell paste & 0,5mL of \ce{Ce}-solution & 1 \\
        %Cell paste & 0,5mL of \ce{Ce}-solution & 2 \\
        Proteins dissolved in water & 1mL of \ce{NdFeB}-magnet, dissolved in water & 0, 1 and 2 \\
        %Proteins dissolved in water & 1mL of \ce{NdFeB}-magnet, dissolved in water & 1 \\
        %Proteins dissolved in water & 1mL of \ce{NdFeB}-magnet, dissolved in water & 2 \\
        Cell paste & 1mL of \ce{NdFeB}-magnet, dissolved in water & 0, 1 and 2 \\
        %Cell paste & 1mL of \ce{NdFeB}-magnet, dissolved in water & 1 \\
        %Cell paste & 1mL of \ce{NdFeB}-magnet, dissolved in water & 2 \\
    \end{tabularx}
    \caption{Treatments given to bacteria cultures before lysis.}
    \label{tab:ir_exp_treatment}
\end{table}

We put a drop of a sample onto the ATR-IR-Spectrometer and measured the IR spectrum.
Before each measurement, we cleaned the ATR crystal with a drop of ethanol on a soft tissue.



\subsection{SDS-PAGE}

\section{Arsenazo-III Assay\authorA}

