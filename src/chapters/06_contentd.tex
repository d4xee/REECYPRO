\chapter{Experimental Section}

In the following chapter, we describe the work we carried out and why we did it.
We begin with methods for cultivating bacteria, which are followed by the extraction of proteins from \emph{M. extorquens}.
Afterwards, we describe how we analyzed the proteins.
In the end, we show you how we did the arsenazo-III assay.


\section{Cultivation of Bacteria\authorB}

Preparation of solid nutrient solution for petri dishes:

\textbf{Materials:}

\begin{itemize}
    \item Peptone 2,5g
    \item Meat Extract 1,5g
    \item Agar 7,5g
    \item \ce{H2O} 500mL
    \item Scale
    \item Autoclave bottle
    \item Spatula
\end{itemize}


\textbf{Execution:}
\begin{itemize}
    \item Weigh all the necessary ingredients
    \item Fill the autoclave bottle with around 100ml of water
    \item Add the dry ingredients to the autoclave bottle
    \item Mix the dry ingredients with water
    \item Add the remaining water to the bottle
    \item Shake until mixed
\end{itemize}


Finalizing the solid nutrient solution to be poured into petri dishes:

\textbf{Materials:}

\begin{itemize}
    \item Petri dishes
    \item Autoclave
    \item Autoclave indicator tape
    \item Sterile workbench
\end{itemize}


\textbf{Execution:}

\begin{itemize}
    \item Add a piece of indicator tape on the cap of the autoclave bottle
    \item Autoclave the nutrient solution at 121°C and 1 bar for 15 minutes
    \item After autoclaving the nutrient solution swiftly pour it into the petri dishes
    \item Leave the nutrient solution to harden
    \item Flip the petri dishes carefully on their cover and put them in the fridge
\end{itemize}


\section{Protein Extraction\authorB}


\section{Protein Analysis\authorA}

\subsection{IR-Spectrometry}

The IR-Spectrometry was done using an ATR-IR-Spectrometer.
This meant that we could place the sample directly onto the device, without additional sample preparation.

The following table gives an overview of the lysed samples of \emph{M. extorquens} we measured, what treatments we gave the bacteria to grow in and when we measured the samples after the treatment.

\begin{table}[H]
    \begin{tabularx}{\textwidth}{l p{5cm} X}
        \hline
        \textbf{Type of sample}      & \textbf{Treatment}                           & \textbf{Weeks of growth after treatment} \\ \hline
        Proteins dissolved in water  & no treatment                                 & 0, 1 and 2                               \\
        %Proteins dissolved in water & no treatment & 1 \\
        %Proteins dissolved in water & no treatment & 2 \\
        Cell paste                   & no treatment                                 & 0, 1 and 2                               \\
        %Cell paste & no treatment & 1 \\
        %Cell paste & no treatment & 2 \\
        Proteins dissolved in water  & 0,5mL of \ce{Ce}-solution                    & 0, 1 and 2                               \\
        %Proteins dissolved in water & 0,5mL of \ce{Ce}-solution & 1 \\
        %Proteins dissolved in water & 0,5mL of \ce{Ce}-solution & 2 \\
        Cell paste                   & 0,5mL of \ce{Ce}-solution                    & 0, 1 and 2                               \\
        %Cell paste & 0,5mL of \ce{Ce}-solution & 1 \\
        %Cell paste & 0,5mL of \ce{Ce}-solution & 2 \\
        Proteins dissolved in water  & 1mL of \ce{NdFeB}-magnet, dissolved in water & 0, 1 and 2                               \\
        %Proteins dissolved in water & 1mL of \ce{NdFeB}-magnet, dissolved in water & 1 \\
        %Proteins dissolved in water & 1mL of \ce{NdFeB}-magnet, dissolved in water & 2 \\
        Cell paste                   & 1mL of \ce{NdFeB}-magnet, dissolved in water & 0, 1 and 2                               \\
        %Cell paste & 1mL of \ce{NdFeB}-magnet, dissolved in water & 1 \\
        %Cell paste & 1mL of \ce{NdFeB}-magnet, dissolved in water & 2 \\
        \hline
    \end{tabularx}
    \caption{Types of samples measured with IR-Spectrometry.}
    \label{tab:ir_exp_treatment}
\end{table}

We put a drop of a sample onto the ATR-IR-Spectrometer and measured the IR spectrum.
Before each measurement, we cleaned the ATR crystal with a drop of ethanol on a soft tissue.

\subsection{SDS-PAGE}

The SDS-PAGE was carried out as it is described in the guide of the manufacturer of our electrophoresis cell, Bio-Rad Laboratories, Inc.~\cite{sdsbulletin}, with some slight modifications.

\subsubsection{Gel Casting:}

\textbf{Materials and Methods:}

\begin{table}[H]
    \begin{tabularx}{\textwidth}{ X l l }
        \hline
        \textbf{Ingredient} & \textbf{Volumina for Stacking Gel} & \textbf{Volumina for Resolving Gel} \\ \hline
        30\% Acrylamide     & 1,485mL                            & 4,5mL                               \\
        0,5M Tris-HCl pH6,8 & 3,78mL                             & -                                   \\
        1,5M Tris-HCl pH8,8 & -                                  & 3,75mL                              \\
        10\% SDS            & 150µL                              & 150µL                               \\
        di\ce{H2O}          & 9mL                                & 5,03mL                              \\
        TEMED               & 18µL                               & 7,5µL                               \\
        10\% APS            & 90µL                               & 75µL                                \\
        \hline
    \end{tabularx}
    \caption{Ingredients for SDS-PAGE gel.}
    \label{tab:ingredients_sdspage}
\end{table}

\begin{itemize}
    \item Mix all ingredients except for TEMED and APS
    \item Add TEMED and APS to the resolving gel
    \item Transfer resolving gel into the gel cassette
    \item Add ethanol on top, to ensure that the stacking gel binds to the resolving gel
    \item Wait until the resolving gel has polymerized
    \item When the resolving gel is firm, pour off the ethanol
    \item Put the gel comb into the gel cassette, so that one end is between the glasses
    \item Add TEMED and APS to the stacking gel
    \item Fill the stacking gel into the gel cassette
    \item Make sure that no air bubble is between the tines of the comb
    \item Press the comb gently into the fluid gel
    \item Wait until the stacking gel has polymerized
\end{itemize}

We increased the amount of TEMED and APS by 20 percent compared to the manufacturer's guide, because oftentimes the resolving gel did not polymerize.

\subsubsection{Running Buffer:}
\textbf{Materials and Methods:}
\begin{itemize}
    \item Tris Base 30,30g
    \item Glycine 144,70g
    \item SDS 10,00g
    \item Mix all ingredients with 1000mL of deionized water
\end{itemize}

\subsubsection{Probe Preparation:}
\textbf{Materials and Methods:}

Laemmli-Buffer:
\begin{itemize}
    \item 0,5M Tris-HCl pH6,8 3,75mL
    \item Glycerol 100\% 7,5mL
    \item 1,0\% Bromophenol blue 0,3mL
    \item 10\% SDS 6,0mL
    \item di\ce{H2O} to 30mL
    \item Mix all ingredients
    \item When the buffer is yellow, add Tris base until it is blue
\end{itemize}

Probe Preparation:
\begin{itemize}
    \item Mix the sample with the same amount of Laemmli-Buffer
    \item Heat the mixture to 95°C for 5min
\end{itemize}

We did not add \(\beta\)-mercaptoethanol to the sample/buffer mixture, because the chemical was not available at that time.

\subsubsection{Performing Gel Electrophoresis:}
\textbf{Materials and Methods:}
\begin{itemize}
    \item Insert the gel(s) with removed comb(s) into the running module
    \item Fill the gel box with the running buffer to around three quarters
    \item Insert 30µL of protein marker into one well
    \item Insert 15µL of sample into the remaining wells
    \item Fill the gel box completely with running buffer
    \item Put the lid onto the box and plug in the electrodes
    \item Let the SDS-PAGE run for 60min at 100V
\end{itemize}

Gel Staining:
\begin{itemize}
    \item ------------------------------------------------------------------------------------------------------------------------------------------------------------------------
    \item Mix all ingredients
    \item Remove the glass front plate gently from the gel
    \item Cut the stacking gel carefully off
    \item Remove the gel cautiously from the spacer plate
    \item Put the gel into the staining solution
    \item Let the gel stain overnight
\end{itemize}

Gel Destaining:
\begin{itemize}
    \item ------------------------------------------------------------------------------------------------------------------------------------------------------------------------
    \item Mix all ingredients
    \item Rinse the stained gel under deionized water
    \item Put the stained gel into the destaining solution
    \item Let the gel destain overnight
    \item After the gel destained, rinse it with deionized water
    \item The gel can be stored in a fridge in deionized water
\end{itemize}

\section{Arsenazo-III Assay\authorA}

