\chapter{Results and Discussion}

\section{Cultivation of Bacteria\authorB}
Upon acquisition of the bacterial strain, it underwent cultivation on agar plates followed by
incubation. However, initial observations within the first week of incubation did not yield
satisfactory results, as Methylorubrum Extorquens (M. Extorquens) failed to produce a visibly
discernible orange culture indicative of successful growth.
By the third week of cultivation, a distinct orange dot appeared on the surface of the incubated
agar plate, signifying the emergence of bacterial growth. The bacteria were meticulously scraped
from the solid nutrient media for subsequent cultivation in liquid nutrient media, thereby
facilitating a transition from solid to liquid growth conditions.
Subsequent growth in liquid media exhibited a remarkable proliferation of bacterial colonies. To
maintain optimal growth conditions and prevent overpopulation-induced stress, a routine
procedure of decanting 75\% of the liquid culture from the flasks and replenishing them with fresh
liquid media was implemented on a weekly basis. This critical step ensured the sustained viability
and productivity of the bacterial population.
Toward the latter stages of the project, flasks were emptied completely, yet bacterial colonies
regenerated solely from residual deposits adhering to the inner walls of the flasks.
It was also observed that the addition of Methanol had a significant impact on M. Extorquens’
growth speeding up it’s growth by 20\%-50\%, this is explainable by M. Extorquens’ Methanol
metabolizing capabilities.

\section{Protein Analysis\authorA}

\section{Arsenazo-III Assay\authorA}

