\chapter{Detection and Measurement of REE concentration\authorA{}}

%Rare Earth Elements (for short: REEs) play a critical role in modern-day life.
%They are used in nearly every device that uses electrical power to operate.
%A few example where REEs are essential are: lasers, computer monitors, electric motors, high-power magnets, liquid crystal displays (LCDs), solar panels~\cite{usageofrees}.
%In this context, it is clear that the demand for REEs is rising rapidly.
%In the following years, with more and more electronic devices produced, most of them will eventually end as electronic waste.
%Recycling REEs from this waste is crucial for the worlds REE supply.
%Current recycling methods are mostly harmful to the environment and very costly~\cite{recyclingcurrent}.
%But new recycling methods have emerged in the last years and one of them, using the technique of biosorption, is the subject of this thesis.
%To understand how this process works, it is important to know the following techniques.

The detection of rare earth elements in a probe is a crucial step in our work.
It allows us to quantify the effectiveness of our process.

In modern chemistry,
quantification of elements in a probe is usually done with inductively coupled plasma mass spectroscopy or ICP-MS\@.
However, as the ICP-MS uses machines which are very, very expensive,
this was not an option as it exceeded our limited financial resources by far.
Instead, we had to search for other methods to detect and quantify rare earths.

In our work, we used two precipitation reactions and one method to quantify the concentration of REEs.


\section{Precipation Reactions}

\subsection{Cer Precipitation Reaction}
The precipitation reaction for cer works by utilizing the oxidation states +III and +IV~\cite{cerdetection,janderblasius}.

Cer in the aforementioned states forms complexes together with \ce{H2O2}.
The complexes are called cer peroxide hydrates, and their chemical formulas are \ce{Ce(OH)2(OOH)} and \ce{Ce(OH)3(OOH)}.
These complexes fall out of the solution as a red-brown colored precipitate.

\subsection{Neodymium Precipitation Reaction}
The reaction to detect neodymium is a bit more complicated.
It also uses the +III oxidation state of neodymium.
The neodymium reacts with acetic acid to form neodymium acetate.
As the last step, iodide is given to the solution which forms a blue-colored complex together with the neodymium acetate~\cite{janderblasius}.

\section{Arsenazo III Assay}
