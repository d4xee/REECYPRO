\chapter{Detection of REEs\authorB{}}

Rare Earth Elements (for short: REEs) play a critical role in modern-day life.
They are used in nearly every device that uses electrical power to operate.
A few example where REEs are essential are: lasers, computer monitors, electric motors, high-power magnets, liquid crystal displays (LCDs), solar panels~\cite{usageofrees}.
In this context, it is clear that the demand for REEs is rising rapidly.
In the following years, with more and more electronic devices produced, most of them will eventually end as electronic waste.
Recycling REEs from this waste is crucial for the worlds REE supply.
Current recycling methods are mostly harmful to the environment and very costly~\cite{recyclingcurrent}.
But new recycling methods have emerged in the last years and one of them, using the technique of biosorption, is the subject of this thesis.
To understand how this process works, it is important to know the following techniques.


\section{Detail in A}

\subsection{Sub-Detail 1 in A}

\subsection{Sub-Detail 2 in A}


\section{Detail in A}
