\chapter{Conclusion\authorA{}}
The process of recycling of rare earths from e-waste using bacteria is a more eco-friendly and energy efficient way than currently established recycling methods.
In our project, we achieved to carry out this process and to determine its efficiency.
Hereby, it is important to know that we only measured the natural capacity of \emph{M. extorquens} without any additional changes.

In brief, our project can be summarized as follows:
We found a way to efficiently recycle rare earth elements from e-waste.
This works with the bacteria \emph{Methylorubrum extorquens}, which has the ability to use rare earth elements in its metabolism.
This property of the bacteria is essential because the e-waste is simply given in a crushed form to the culture medium.
The rare earths accumulate naturally in the bacteria.
The bacteria can then be opened to recover the rare earths.

\textbf{<Beschreibung von Ergebnisse / Effizienz>}

We learned a lot during the time of this project, because neither of us had previous knowledge in the field of microbiology.
This meant that we had to research everything from the ground up.
In the beginning, we thought we would do a lot of things differently than we do now.
But after three months of work, we came to a dead end because our school lacked the required equipment.
This had the consequence that we had to pivot our work in a new direction.
Afterward, we finally managed to achieve results.

The key method, which we discovered late in the project was the arsenazo-III assay.
This assay is a method to determine the concentration of rare earth elements in a sample.
Without this method, we would not have achieved any results at all, because all the other methods we tried did not work well enough.

What is also noteworthy is that we learned that at any given time something unexpected can happen, which ruins the work of a whole day.
