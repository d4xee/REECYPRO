\chapter*{Abstract}
\addcontentsline{toc}{chapter}{Abstract}

The rare earth elements (REEs) are metals, which are used in nearly every electronic device.
The demand for new REEs is steadily rising.
Yearly, millions of tonnes of electronic waste (e-waste) are generated.
The e-waste contains valuable metals, but most of them are not recycled.

The recycling of REEs is a challenging process that currently involves the usage of a lot of energy.
But also the mining of new REEs is not eco-friendly.
It involves the usage of strong acids that harm the environment and mining workers.

In recent years, a newly discovered protein found in a specific bacteria was found to be able to tackle the challenge of REE recycling.
The protein can take up the REEs from e-waste, like washing detergents wash dirt out of clothing.

Without any additional preparation of the e-waste other than dust the e-waste, the bacteria is capable of gathering more than 70\% of REEs out of the waste.

This could be used in the near future to recycle REEs in an eco-friendly, climate-friendly and energy-saving way.

\chapter*{Introduction}
\addcontentsline{toc}{chapter}{Introduction}

In 2021, around five million tonnes of electronic waste were generated in the EU alone, but less than 40 percent were recycled.
This waste often contains valuable metals, but currently most of them are disposed.
Some of these disposed metals are the so-called rare earth elements.
The rare earths are critical for every electronic device but they are only used in small quantities, so that conventional recycling is not a economically feasible possibility.

For every new smartphone, for example, new rare earths must be mined.
This happens mostly in countries where compliance with human and environmental rights are questionable.
The following refining of the rare earths is a very energy-consuming, environmentally harmful and climate damaging process.
For a single tonne of neodymium, the most used rare earth element, some 75 tonnes of \ce{CO2} are emitted.
But the problems do not stop there.
There are only a few places on earth were rare earths are mined, because it is mostly not economically viable because of China, who dominates the market.

The largest mine for rare earths is located in China and additionally, China is the largest producer of refined rare earths, which is then used elsewhere to produce electronics.
This means that the world's current supply of rare earth elements is largely monopolized by a country, which does not adhere to human and environmental rights.

A solution could be the recycling of rare earths from electronic waste.
However, currently established methods are either very expensive, damaging to the environment or use a lot of energy.

A promising alternative could be the usage of bacteria to recover the rare earths.
Bacteria have the advantages that they do not need a lot of energy to grow and they only need inexpensive resources to be able to grow.

In this thesis, we tested if, and how rare earth elements can be recycled from electronic waste using bacteria.
We outline how this process works and we report our findings from the actual realisation of this process.