\chapter*{Abstract}
\addcontentsline{toc}{chapter}{Abstract}

Lorem ipsum dolor sit amet, consectetur adipiscing elit. Aenean viverra eget sapien in fringilla. Proin ac neque non lectus vehicula laoreet in cursus enim. Donec et erat ut erat commodo viverra vitae sed risus. Etiam tortor justo, placerat in turpis sit amet, egestas tristique libero. Phasellus metus arcu, viverra at interdum ac, convallis non urna. Sed nunc libero, elementum quis ultricies at, vestibulum in arcu. Nam ultrices felis ut sagittis hendrerit. Vivamus massa sapien, interdum nec dui ac, consectetur venenatis dolor. Integer enim felis, finibus at efficitur eget, viverra vitae purus. Curabitur at libero pretium, vestibulum lacus at, eleifend nisl.

Nullam ut magna quis ante gravida aliquet. Integer ultricies libero vitae quam mollis, non tincidunt justo posuere. Mauris ultricies varius orci non tempus. Sed at ex maximus, tempor libero id, convallis ligula. Donec posuere massa sit amet porttitor vehicula. Donec porttitor luctus dui sed blandit. Ut egestas, enim id egestas auctor, est ligula accumsan diam, nec lacinia massa elit vitae purus.

Ut consectetur ipsum id nisl sodales varius. Pellentesque habitant morbi tristique senectus et netus et malesuada fames ac turpis egestas. Aliquam venenatis varius maximus. Aenean aliquet mi a magna tempor, et sagittis ligula tincidunt. Maecenas ornare non leo et dignissim. Nunc ac feugiat magna. Nulla at sollicitudin massa, nec sollicitudin libero. Nunc posuere dolor mauris, non congue neque lobortis eget. Vestibulum ex leo, ullamcorper quis malesuada in, maximus quis nisl. Morbi neque diam, dignissim non suscipit ac, molestie at sem. In hac habitasse platea dictumst. Curabitur dictum eros non ipsum luctus, a malesuada sapien iaculis. Nam mauris nisi, sodales et consectetur quis, varius eu lacus.

\chapter*{Introduction}
\addcontentsline{toc}{chapter}{Introduction}

In 2021, around five million tonnes of electronic waste were generated in the EU alone, but less than 40 per cent were recycled.
This waste often contains valuable metals, but currently most of them are disposed.
Some of these disposed metals are the so-called rare earth elements.
The rare earths are critical for every electronic device but they are only used in small quantities, so that conventional recycling is not a economically feasible possibility.

For every new smartphone, for example, new rare earths must be mined.
This happens mostly in countries where compliance with human and environmental rights are questionable.
The following refining of the rare earths is a very energy-consuming, environmentally harmful and climate damaging process.
For a single tonne of neodymium, the most used rare earth element, some 75 tonnes of \ce{CO2} are emitted.
But the problems do not stop there.
There are only a few places on earth were rare earths are mined, because it is mostly not economically viable because of China, who dominates the market.

The largest mine for rare earths is located in China and additionally, China is the largest producer of refined rare earths, which is then used elsewhere to produce electronics.
This means that the world's current supply of rare earth elements is largely monopolized by a country, which does not adhere to human and environmental rights.

A solution could be the recycling of rare earths from electronic waste.
However, currently established methods are either very expensive, damaging to the environment or use a lot of energy.

A promising alternative could be the usage of bacteria to recover the rare earths.
Bacteria have the advantages that they do not need a lot of energy to grow and they only need inexpensive resources to be able to grow.

In this thesis, we tested if, and how rare earth elements can be recycled from electronic waste using bacteria.
We outline how this process works and we report our findings from the actual realisation of this process.