\chapter{System Overview}

In order to understand the process of the recovery of rare earth elements from electronic waste with biosorption, the key procedures and techniques are described briefly in the following section.


\section{Detection of REEs\authorA}

\subsection{Precipation Reactions}
A relatively simple proof if a probe contains REEs is a precipitation reaction.
It works by utilizing the +III and the +IV oxidization states of the REEs.
These are used to form complexes with other molecules which express themselves as a coloured precipitation in the probe solution \footnote{Jander/Blasius: "Lehrbuch der analytischen und präparativen anorganischen Chemie", Chapter 4.3.3.10}.
As an example, a Ce precipitation reaction is shown in~\ref{fig:cer_precipitation_cropped} with an orange-red precipitate.

\begin{figure}[H]
    \centering
    \includegraphics[width=0.75\textwidth]{./media/images/ree_precipitation_reaction_cropped}
    \caption{Precipitations of a successful REE detection reaction. The test tube on the righthandside does not show any precipitation because the probe was deionized water.}
    \label{fig:cer_precipitation_cropped}
\end{figure}

However, you must be careful, because of the REEs chemical similarity, the detection of a specific REE is not always possible with these precipitation methods.

\subsection{Arsenazo III Assay}

\section{Bacteria\authorB{}}

\subsection{Methylorubrum extorquens}

\subsection{Cultivation}

\subsection{Lanmodulin\authorA}

Lanmodulin (LanM) is a protein that is produced by \textit{M. extorquens}, a lanthanide-utilizing bacteria~\cite{lanmdiscovery}.
LanM is not essential for the growth or survival of \textit{M. extorquens}, and it is only produced when the bacteria are in a medium with presence of \ce{Ln^{III}} or \ce{Ce^{III}} ions~\cite{lanmroleinbiology}.
However, the mechanisms that include LanM are not understood as a whole to this day.

\begin{figure}[H]
    \centering
    \includegraphics[width=0.6\textwidth]{./media/images/lanm_structure}
    \caption{Graphical visualisation of the structure of lanmodulin. The EF-hands are indicated by EF, this is where the REEs can bind to the protein. In this visualisation the turquoise coloured spheres are \ce{Y^{III}} ions which are bound to the EF-hands. Picture from "The biochemistry of lanthanide acquisition, trafficking and utilization", Emily R. Featherston and Joseph A. Cotruvo \cite{lanmroleinbiology}.}
    \label{fig:lanm_structure}
\end{figure}

The most important characteristic of LanM is, that the molecule is able to bind lanthanide ions, primarily light REEs (LREEs).
When LanM does this, it undergoes a transformation from a disordered state to a compact form of itself.
The REEs are hereby bound to the so-called EF-hands which favour to bind to \ce{Ln^{III}} and other lanthanoids over \ce{Ca^{II}} which is usually associated with these EF-hands~\cite{lanmstructure}.


\section{Protein Extraction/IR-Spectrometry\authorB{}}

\subsection{Cell Lysis}

\subsection{SDS-PAGE}
