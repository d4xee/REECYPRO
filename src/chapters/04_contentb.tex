\section{Methylorubrum extorquens}


\subsection{Taxonomy\authorB{}}

\subsubsection{Phylum Pseudomonia}
\emph{Pseudomonadota} is a major phylum of gram-negative bacteria (information about gram-negative bacteria will follow further down).
They are incredibly diverse, encompassing
pathogens, free-living species, nitrogen-fixing bacteria, and many more.
\emph{Pseudomonadota} exhibit a large range of shapes and sizes as well as metabolisms and
habitats which will also be discussed further down.
The diversity of \emph{Pseudomonadota} makes them play a major role in the world's nutrient cycling ranging from crucial ecological relationships with humans to simple things such as nitrogen fixation~\cite{pseudomonadota}.
\emph{Pseudomonadota} includes five classes but only the class \emph{Alphaproteobacteria} is of importance for us.

\subsubsection{Class Alphaprotoebacteria}
\emph{Alphaproteobacteria} is a highly diverse class of bacteria belonging to the phylum \emph{Pseudomonadota}.
They are named after the first letter of the Greek alphabet (alpha) due to being one of the first major lineages to diverge within the \emph{Proteobacteria} phylum.

This class is incredibly varied, encompassing bacteria with a range of lifestyles including phototrophs (light-using), methanotrophs (methane-utilizing), symbionts (mutually beneficial relationships with other organisms), and pathogens (disease-causing).

Soil, water including cold deep-sea vents, hot springs, and symbiotic relationships even
with humans are natural habitats of \emph{Proteobacteria}~\cite{gammaproteobacteria}.

\textbf{Rhizobium:} These bacteria form a symbiotic partnership with legumes, such as peas and soybeans.
\emph{Rhizobium} colonizes the legume's root nodules and fixes atmospheric nitrogen into a usable form that is essential for plant growth.

\textbf{Wolbachia:} This widespread genus of bacteria lives symbiotically within insects and other arthropods.
\emph{Wolbachia} can manipulate the host's reproduction in various ways, sometimes even influencing sex ratios or protecting the host from viruses.

\textbf{Rickettsia:} This genus includes several species that are obligate intracellular pathogens, meaning they can only live and reproduce inside the cells of a host organism.
\emph{Rickettsiae} causes various human diseases, including typhus fever and Rocky Mountain spotted fever.

\textbf{Magnetococcus:} These magnetotactic bacteria contain magnetosomes, specialized
organelles that allow them to align and move along magnetic fields.

\subsubsection{Order Hyphomicrobiales}
\emph{Hyphomicrobiales} can utilize single-carbon compounds like methanol as an energy
source, the bacterium \emph{Methylorubrum} \emph{extorquens} does this, for example.

\emph{Hyphomicrobiales} produce carotenoid pigments and therefore appear pink or orange in colonies.
These colonies are aerobic, which means they require oxygen for growth.
They inhabit a large variety of environments including soils, plant surfaces, root structures, water and dust.

They also play important ecological roles in their habitats, like plant-microbe interactions when metabolizing methanol on plant leaves or carbon and nitrogen cycling in various environments~\cite{methylobacteria_groups}.

\subsubsection{Genus Methylorubrum}
They use specialized pathways to break down methanol for energy and to create biomass.
This metabolic capability has potential applications in Bioremediation, which
means that this bacteria can clean up methanol-contaminated areas.
This family of bacteria is also able to produce valuable chemicals from methanol~\cite{new_methylorubrum}.

Bacteria of the genus \emph{Methylorubrum} are rod-shaped or slightly bent and show pink or
orange pigmentation like every genus that belongs to the order \emph{Methylobacterium.}

\begin{figure}[H]
    \centering
    \includegraphics[width=0.9\textwidth]{./media/images/mextorquens_on_leaf}
    \caption{Pink \emph{Methylorubrum extorquens} on a leaf utilizing the plant's nutritiens.}
    \label{fig:mextorquens_on_leaf}
\end{figure}

\subsubsection{Species Extorquens}
In our thesis, the \emph{extorquens} bacterium species holds immense significance as it displays all the key characteristics of the aforementioned groups to which it belongs.
The \emph{Methylorubrum extorquens} strain is unique in its ability to utilize methanol or methane as its sole source of carbon and energy.
Additionally, this bacterium has the capability to metabolize various compounds such as acetate, pyruvate, and succinate, which are converted to energy.
This makes the \emph{Methylorubrum extorquens} strain a particularly fascinating subject for further research and analysis.

\begin{figure}[H]
    \centering
    \includegraphics[width=0.8\textwidth]{./media/images/mextorquens_sealed}
    \caption{\emph{M. extorquens} in a sealed petri dish.}
    \label{fig:mextorquens_petri_sealed}
\end{figure}


\subsection{Methanol Metabolism\authorB}
\emph{Methylorubrum extorquens} exhibits the ability to utilize the simple alcohol methanol \ce{CH3OH} as its only source of carbon and energy.
This metabolism is explained in three steps:

\begin{enumerate}
    \item \textbf{Initiation: Oxidation of Methanol}
    \begin{itemize}
        \item Location: Periplasm (the space between the inner and outer cell membranes)
        \item Enzymes:
        \begin{itemize}
            \item Methanol dehydrogenase (MposX):
            \begin{itemize}
                \item XoxF1: Requires lanthanides for activity, oxidizing methanol to
                formaldehyde (\ce{HCHO}) and releasing \ce{H+}.
                \item XoxF2: Less dependent on lanthanides, potentially involved in
                regulating methanol uptake.
            \end{itemize}
        \end{itemize}
        \item Importance: Formaldehyde is a toxic intermediate, requiring rapid conversion for M. extorquens' survival~\cite{methanol_metabolism}.
    \end{itemize}
    \begin{figure}[H]
        \centering
        \includegraphics[width=0.9\textwidth]{./media/images/mextorquens_metabolism_methanol}
        \caption{Schematic of the metabolic processes to oxidizing methanol to formaldehyde which is reduced or eliminated and used for growth by cells.}
        \label{fig:mextorquens_metabolism_methanol}
    \end{figure}
    \item \textbf{Capturing the Essence: Fixation of Formaldehyde}
    \begin{itemize}
        \item Molecule: Dephosphotetrahydromethanopterin (\ce{dH4MPT}) acts as a one-carbon
        carrier.
        \item Enzyme: Formaldehyde-activating enzyme (Fae) catalyzes the reaction, attaching
        formaldehyde to \ce{dH4MPT}\@.
        \item Significance: Enables the transport of formaldehyde into the cytoplasm for
        further metabolism~\cite{methanol_metabolism}.
    \end{itemize}
    \item \textbf{Carbon Assimilation: The Serine Cycle Takes Over}
    \begin{itemize}
        \item Location: Cytoplasm
        \item Pathway:
        \begin{enumerate}
            \item Formate dehydrogenase:
            Oxidizes the formaldehyde-\ce{dH4MPT} complex, generating formate (\ce{HCOO}).
            \item Formate acetyltransferase:
            Condenses formate with acetyl-CoA, forming S-acetyl-CoA\@.
            \item Serine hydroxymethyltransferase: Transfers the one-carbon unit from Sacetyl-CoA to glycine, forming serine.
            \item Serine transaminase: Converts serine to pyruvate, a key metabolic
            intermediate.
        \end{enumerate}
        \item Importance: The serine cycle efficiently converts the one-carbon unit from
        methanol into usable cellular building blocks~\cite{methanol_metabolism}.
    \end{itemize}
    \begin{figure}[H]
        \centering
        \includegraphics[width=0.9\textwidth]{./media/images/mextorquens_metabolizing_methanol}
        \caption{\emph{Methylorubrum extorquens} metabolizing methanol}
        \label{fig:mextorquens_metabolizing_methanol}
    \end{figure}
\end{enumerate}


\subsection{Lanmodulin\authorA}
Lanmodulin (LanM) is a protein that was discovered in 2018 in the bacteria \emph{Methylorubrum extorquens}~\cite{lanmdiscovery}.
The molecule is around 12kDa in size, and it possesses unique properties, even when compared to other similar proteins.
Lanmodulin contains four of the so-called EF-hands.
These hands are normally used to sense \ce{Ca^{II}} ions.
Lanmodulin, however, is able to bind \ce{Ln^{III}} and other lanthanide ions (which most of the rare earth elements belong to) to this EF-Hands, not only \ce{Ca^{II}}~\cite{lanmstructure}.

\begin{figure}[H]
    \centering
    \includegraphics[width=0.6\textwidth]{./media/images/lanm_structure}
    \caption{Graphical visualization of Lanmodulins structure. EF indicates the EF-hands, this is where the REEs can bind to the protein. In this visualization, the turquoise-colored spheres are \ce{Y^{III}} ions which are bound to the EF-hands. Picture from "The biochemistry of lanthanide acquisition, trafficking and utilization", Emily R. Featherston and Joseph A. Cotruvo \cite{lanmroleinbiology}.}
    \label{fig:lanm_structure2}
\end{figure}

The second interesting property originates from the ability to bind lanthanide ions.
It was found that LanM does not only bind lanthanide ions, but it even favors them to bind to its EF-hands.
The affinity for the lanthanides is around \(10^{8}\) times higher than for \ce{Ca^{II}}.
This means that, in a solution with, for example, \ce{Ln^{III}} and \ce{Ca^{II}} ions, only very few calcium ions will bind to the LanM.

When LanM binds the lantanide ions, something interesting happens: it undergoes a transformation.
It changes its shape and morphs into a sphere-like structure, which contains the ions inside.
However, how and why exactly lanmodulin does this, is the subject of ongoing research~\cite{lanmongoingresearch}.

\subsection{Growth\authorB}
\emph{Methylorubrum extorquens} thrives at temperatures between 30°C and 35°C, making it a
mesophilic bacteria.
To promote its optimal growth, the bacteria was placed in a swivel
incubator set to this temperature.
Additionally, the nutrient solution needs to be slightly acidic to neutral, with a pH range of 6.5-7.5, to further enhance growth.
Because \emph{M. extorquens} is an aerobic bacteria, the solution in which it is cultivated must be able to exchange gas and absorb oxygen.
This is achieved by sealing the Erlenmeyer flask with a piece of sterile cotton that allows oxygen to pass through while keeping other bacteria and fungi out.

\begin{figure}[H]
    \centering
    \includegraphics[width=0.9\textwidth]{./media/images/swivel_incubator}
    \caption{Swivel incubator with temperature control used for cultivating \emph{M. extorquens}.}
    \label{fig:swivel_incubator}
\end{figure}

Under optimal conditions, the exponential growth phase of \emph{M. extorquens} typically lasts six to eight hours, during which the number of cells increases rapidly.
However, this growth phase comes to a halt due to a lack of nutrients or waste product accumulation, resulting in the stationary phase.
After this point, the viability and number of cells gradually decrease, known as the death phase.


\subsection{Lysis\authorB}\label{sec:me_lysis}
In order to obtain Lanmodulin and REEs, it is necessary to break open the cell walls of
the bacteria.
This process, known as lysis, can be accomplished using a variety of
techniques - either mechanical or enzymatic.
Mechanical methods include bead beating, French press lysis, and shock freezing, while enzymatic lysis can be achieved through lysozyme treatment, which utilizes an enzyme that specifically breaks down bacterial cell walls.
For this project, a combination of shock freezing and cell wall disruption using an ultrasonic bath was selected.